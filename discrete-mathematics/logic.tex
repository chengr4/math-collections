\documentclass[12pt,a4paper]{article}
\usepackage{fullpage}
\usepackage{amsfonts, amssymb, amsmath}
\usepackage{parskip} % skip a line instead of indenting

\title{Logic}
\author{R4 Cheng}
\date{}

\begin{document}
\maketitle

De Morgan's laws:
\begin{itemize}
  \item $\neg (p \lor q) \equiv \neg p \land \neg q $
  \item $\neg (p \land q) \equiv \neg p \lor \neg q $
\end{itemize}

Usage:

De Morgan's laws allow us to negate complex logical statements and to rewrite them in a different form that may be easier to understand or manipulate

\subsection*{Implication $p \Rightarrow q$}
Def: An implication $p \Rightarrow q$ is a true if $p$ is false or $q$ is true.

\begin{tabular}{|c|c|c|}
\hline
$p$ & $q$ & $p \Rightarrow q$ \\
\hline
T & T & T \\
T & F & F \\
F & T & T \\
F & F & T \\
\hline
\end{tabular}

\texttt{> "innocent until proven guilty" }

\begin{itemize}
  \item $(q \Rightarrow p)$: converse of $p \Rightarrow q$
  \item $(\neg p \Rightarrow \neg q)$: contrapositive of $p \Rightarrow q$
  \item $p \Rightarrow q \equiv \neg q \Rightarrow \neg p \equiv (\neg p) \lor q $
\end{itemize}

\subsection*{If and only if $p \Leftrightarrow q$}

$$ p \Leftrightarrow q \equiv (p \Rightarrow q) \land (q \Rightarrow p) $$

\subsection*{Quantifiers}

\begin{itemize}
  \item $\neg (\forall x, p(x)) \equiv \exists x, \neg p(x) $
  \item $\neg (\exists x, p(x)) \equiv \forall x, \neg p(x) $
\end{itemize}

\end{document}
