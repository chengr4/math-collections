\documentclass[12pt,a4paper]{article}
\usepackage{amsfonts, amssymb, amsmath}
\usepackage{fullpage}
\usepackage{parskip} % skip a line instead of indenting
% \usepackage{graphicx} % for inserting images
\usepackage{amsthm}
\usepackage{xcolor}
\usepackage{tikz} % for plot

\title{Engineering Mathematics}
\author{R4 Cheng}
\date{\today}

\newcommand{\remark}[1]{
    $>$ {\color{blue} #1}
}

\begin{document}
\maketitle

\section*{Common Reasoning}

\[\int \frac{1}{x} \, dx = \ln|x| + C\]

\remark{This is valid for $x \neq 0$, and the absolute value ensures the logarithm is defined for both positive and negative values of $x$.}

\section*{Total Derivative}

\[ d\phi = \frac{\partial \phi}{\partial x}dx + \frac{\partial \phi}{\partial y}dy\]

\remark{When partially differentiating with respect to $x$, treat $y$ as a constant, and vice versa.}

\section*{Linear Algebra}

\subsection*{Transportation}

\[
A = 
\begin{bmatrix}
  1 & 2 & 3 \\
  4 & 5 & 6 \\
  7 & 8 & 9
\end{bmatrix}
\Rightarrow
A^T =
\begin{bmatrix}
  1 & 4 & 7 \\
  2 & 5 & 8 \\
  3 & 6 & 9
\end{bmatrix}
\Rightarrow
A^T_{ij} = A_{ji}
\]

\subsection*{Speical types of Matrices}

\subsubsection*{Diagonal Matrix}

E.g.

\[
D =
\begin{bmatrix}
  1 & 0 & 0 \\
  0 & 2 & 0 \\
  0 & 0 & 3
\end{bmatrix}
\]

Another example ($3 \times 4$):

\[
D =
\begin{bmatrix}
  5 & 0 & 0 & 0 \\
  0 & 1 & 0 & 0 \\
  0 & 0 & 7 & 0
\end{bmatrix}
\]

\subsubsection*{Identity Matrix (Unit Matrix)}

\[
I =
\begin{bmatrix}
    1 & 0 & 0 \\
    0 & 1 & 0 \\
    0 & 0 & 1
\end{bmatrix}
\]

\end{document}