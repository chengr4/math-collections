\documentclass[12pt,a4paper]{article}
\usepackage{amsfonts, amssymb, amsmath}
\usepackage{parskip} % skip a line instead of indenting

\begin{document}

The distributive property states that $a(b+c)=ab+ac$, for all $a, b, c \in \mathbb{R}$. \\[12pt]
The equivalence class of $a$ is $[a]$

% curly brackets
\{ \}

% dollar sign
This is a dollar sign: $\$11.50$

% parentheses without left and right (bad)
$$2(\frac{1}{x^2-1})$$
% parentheses with left and right
$$2\left(\frac{1}{x^2-1}\right)$$
% square brackets
$$2\left[\frac{1}{x^2-1}\right]$$
% curly brackets
$$2\left\{\frac{1}{x^2-1}\right\}$$
% angular brackets
$$2\left\langle\frac{1}{x^2-1}\right\rangle$$
% absolute value
$$2\left|\frac{1}{x^2-1}\right|$$
\end{document}