\documentclass[12pt]{article}
\usepackage{amsfonts, amssymb, amsmath}
\usepackage{fullpage}
\usepackage{parskip} % skip a line instead of indenting

\begin{document}

This is the symbol for the set of all real numbers: $\mathbb{R}$

This is the symbol for the set of all integers: $\mathbb{Z}$

This is the symbol for the set of all rational numbers: $\mathbb{Q}$

% a0, a1, a2, ..., a100
$${a_{0}, a_{1}, a_{2}, \ldots, a_{100}}$$
$${y=\sin{x}}$$
$${y=\csc{\theta}}$$
$${y=\cos^{-1}{x}}$$

Log functions:
$${\log{x}}$$

Fractions:

$${\frac{1}{2}}$$
About $\displaystyle \frac{2}{3}$ of the glass is full \\[12pt]
% to compare
About $\frac{2}{3}$ of the glass is full

Calculus:

$$\left.\frac{dy}{dx}\right|_{x=1}$$
$$\lim_{x \to \infty} f(x)$$
$$\lim_{x \to a} \frac{f(x)-f(a)}{x-a}$$

% integral
$$\int_{a}^{b} f(x) dx$$
$$\int \sin{x} dx = -\cos{x} + C$$
$$displaystyle{\int_{a}^{b} x^2 dx = \left[\frac{x^3}{3}\right]_{a}^{b}}$$

Align:

% basic
\begin{align}
 5x^2-9 = x+3 \\
 5x^2-9x+4 = 0
\end{align}

% align with "=" use "&"
% asterisk to remove numbering
\begin{align*}
 5x^2-9 &= x+3 \\
 5x^2-9x+4 &= 0
\end{align*}

\begin{align}
  5x^2-9 = x+3 \\
  5x^2-9x+4 = 0
\end{align}

Sum:
$$\sum_{i=1}^{n} i = \frac{n(n+1)}{2}$$

Vector:
$$\vec{v} = \langle 1, 2, 3 \rangle$$

\end{document}