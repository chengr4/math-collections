\documentclass[12pt,a4paper]{article}
\usepackage{fullpage}
\usepackage{amsfonts, amssymb, amsmath}
\usepackage{parskip} % skip a line instead of indenting

\begin{document}
\subsection*{Implication $p \Rightarrow q$}
Def: An implication $p \Rightarrow q$ is a true if $p$ is false or $q$ is true.

\begin{tabular}{|c|c|c|}
\hline
$p$ & $q$ & $p \Rightarrow q$ \\
\hline
T & T & T \\
T & F & F \\
F & T & T \\
F & F & T \\
\hline
\end{tabular}

\texttt{> Did not figure out yet "Vacuous Truth" }


\subsection*{If and only if $p \Leftrightarrow q$}
{TODO}
\subsection*{Axiom}
Def: An axiom is a proposition that is "assumed" to be true

\texttt{> Axioms can be true in some fields, while false in others.}

Axioms should be: 1. consistent 2. complete

Def: A set of axioms is \textbf{consistent} if no proposition can be proved to be both true and false.

Def: A set of axioms is \textbf{complete} if it can be used to prove every proposition is either true or false.

\subsection*{Order of a differential equation}

The \textbf{order} of a differential equation is the highest derivative that appears in the equation.

E.g. the order of $y'' + y' + y = 0$ is 2, because the highest derivative is $y''$.

\[d^2y/dx^2 \Rightarrow \text{2nd order ODE}\]

\subsection*{Degree of a differential equation}

The \textbf{degree} of a differential equation is the power of the highest derivative that appears in the equation.

\[(\frac{d^2y}{dx^2})^1 + (\frac{dy}{dx})^3 = 8 \Rightarrow \text{degree 1st ODE}\]


\subsection*{Linear of differential equations}

In D.E., both unknown function and its derivatives satisfy:

\begin{enumerate}
    \item order is one ($y^{1/2}$ is not linear)
    \item no times ($y^2$, $y \cdot y'$ are not linear)
    \item no non-linear functions ($\sin(y)$, $\ln(y)$ are not linear)
\end{enumerate}

\end{document}
